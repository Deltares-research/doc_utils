% Pandoc compatibility commands
\providecommand{\tightlist}{%
  \setlength{\itemsep}{0pt}\setlength{\parskip}{0pt}%
}
\providecommand{\passthrough}[1]{#1}

%------------------------------------------------------------------------------
% Pandoc compatibility commands
%------------------------------------------------------------------------------
\providecommand{\tightlist}{%
  \setlength{\itemsep}{0pt}\setlength{\parskip}{0pt}%
}
\providecommand{\passthrough}[1]{#1}


%------------------------------------------------------------------------------
% Content (no document structure)
%------------------------------------------------------------------------------
\section{Installation Guide using the Zip
File}\label{installation-guide-using-the-zip-file}

This guide explains how to install the CIP-tool from a zip file
containing all necessary components. This method is for users who
received the complete package (zip file).

\subsection{Table of Contents}\label{table-of-contents}

\begin{enumerate}
\def\labelenumi{\arabic{enumi}.}
\tightlist
\item
  \hyperref[whats-included]{What's Included}
\item
  \hyperref[prerequisites]{Prerequisites}
\item
  \hyperref[installation-steps]{Installation Steps}
\item
  \hyperref[configuration]{Configuration}
\item
  \hyperref[verifying-installation]{Verifying Installation}
\item
  \hyperref[next-steps]{Next Steps}
\end{enumerate}

\begin{center}\rule{0.5\linewidth}{0.5pt}\end{center}

\subsection{What's Included}\label{whats-included}

The zip file contains:

\begin{itemize}
\tightlist
\item
  \textbf{Wheel file} (\passthrough{\lstinline!.whl!}) - The CIP-tool
  Python package
\item
  \textbf{bin/} - External dependencies (RGWM and DIMR runners)

  \begin{itemize}
  \tightlist
  \item
    \passthrough{\lstinline!rgwm\_versie2.4.3/!} - Randvoorwaarden
    Generator Water Modellen
  \item
    \passthrough{\lstinline!dimr\_runners/sobek\_dimr/!} - DIMR with
    Sobek support
  \item
    \passthrough{\lstinline!SingleRunner/!} - SingleRunner wheel (if
    applicable)
  \end{itemize}
\item
  \textbf{examples/} - Example configuration and input files

  \begin{itemize}
  \tightlist
  \item
    \passthrough{\lstinline!config.yaml!} - Pre-configured example
  \item
    \passthrough{\lstinline!inputs/!} - Sample input files
  \item
    \passthrough{\lstinline!test\_model/!} - Example model for testing
  \end{itemize}
\end{itemize}

\begin{center}\rule{0.5\linewidth}{0.5pt}\end{center}

\subsection{Prerequisites}\label{prerequisites}

Before installing CIP-tool, ensure you have:

\begin{itemize}
\tightlist
\item
  \textbf{Python 3.11.x} - The tool requires Python version 3.11
\item
  \textbf{pip} - Python package installer (usually included with Python)
\item
  \textbf{Windows OS} - Required for running Sobek simulations
\end{itemize}

\subsubsection{Check Your Python
Version}\label{check-your-python-version}

\begin{lstlisting}
python --version
\end{lstlisting}

You should see output like: \passthrough{\lstinline!Python 3.11.x!}

If you don't have Python 3.11, you'll need to install it from the
official Python installer or contact your system administrator.

\subsubsection{Install with Conda
(Optional)}\label{install-with-conda-optional}

If you prefer using Conda for Python environment management, you can
create a dedicated environment for CIP-tool with Python 3.11:

\begin{enumerate}
\def\labelenumi{\arabic{enumi}.}
\item
  \textbf{Create a new conda environment:}

\begin{lstlisting}
conda create -n cip-tool python=3.11
\end{lstlisting}
\item
  \textbf{Activate the environment:}

\begin{lstlisting}
conda activate cip-tool
\end{lstlisting}
\item
  \textbf{Verify the Python version:}

\begin{lstlisting}
python --version
\end{lstlisting}
\end{enumerate}

Once activated, you can proceed with the installation steps below using
this conda environment. Remember to activate the environment each time
you want to use CIP-tool.

\begin{center}\rule{0.5\linewidth}{0.5pt}\end{center}

\subsection{Installation Steps}\label{installation-steps}

\subsubsection{Step 1: Extract the Zip
File}\label{step-1-extract-the-zip-file}

\begin{enumerate}
\def\labelenumi{\arabic{enumi}.}
\item
  Locate the downloaded zip file (e.g.,
  \passthrough{\lstinline!CIP-tool-offline-v1.1.0.zip!})
\item
  Extract it to a location on your system, for example:

\begin{lstlisting}
C:\tools\CIP-tool\
\end{lstlisting}
\end{enumerate}

After extraction, your directory should look like:

\begin{lstlisting}
C:\tools\CIP-tool\
|-- CIP_tool-1.1.0-py3-none-any.whl
|-- bin/
|   |-- dimr_runners/
|   |   `-- sobek_dimr/
|   |       `-- x64/
|   |-- rgwm_versie2.4.3/
|   |   |-- bin/
|   |   |   `-- rgwm.exe
|   |   |-- data/
|   |   `-- ...
|   `-- SingleRunner/
|       `-- singlerunner-2.0.1-py3-none-any.whl
`-- examples/
    |-- config.yaml
    |-- inputs/
    `-- test_model/
\end{lstlisting}

\subsubsection{Step 2: Install the
Wheel}\label{step-2-install-the-wheel}

Open PowerShell or Command Prompt and navigate to the extraction
directory:

\begin{lstlisting}
cd C:\tools\CIP-tool
\end{lstlisting}

Install the CIP-tool wheel using pip:

\begin{lstlisting}
pip install CIP_tool-1.1.0-py3-none-any.whl
\end{lstlisting}

\begin{quote}
\textbf{Note:} Replace \passthrough{\lstinline!1.1.0!} with the actual
version number in your wheel file name.
\end{quote}

\subsubsection{Step 3: Verify
Installation}\label{step-3-verify-installation}

Check that the installation was successful:

\begin{lstlisting}
cip-tool --version
\end{lstlisting}

You should see the installed version number.

\begin{center}\rule{0.5\linewidth}{0.5pt}\end{center}

\subsection{Configuration}\label{configuration}

\subsubsection{Update Configuration
File}\label{update-configuration-file}

Edit the \passthrough{\lstinline!config.yaml!} file to point to the
correct paths for the external tools.

If you extracted to \passthrough{\lstinline!C:\\tools\\CIP-tool\\!},
your \passthrough{\lstinline!config.yaml!} should look like:

\begin{lstlisting}
DIMR:
    path: C:/tools/CIP-tool/bin/dimr_runners/sobek_dimr/x64/dimr/scripts

RGWM:
    path: C:/tools/CIP-tool/bin/rgwm_versie2.4.3/bin/rgwm.exe
\end{lstlisting}

\begin{quote}
\textbf{Important:} - Use forward slashes (\passthrough{\lstinline!/!})
in paths, even on Windows, or use escaped backslashes
(\passthrough{\lstinline!\\\\!}) - Adjust the paths if you extracted to
a different location
\end{quote}

\subsubsection{Verify Tool Paths}\label{verify-tool-paths}

Check that the paths in your config file are correct (open a powershell
and type the following commands):

\begin{enumerate}
\def\labelenumi{\arabic{enumi}.}
\item
  \textbf{Verify RGWM executable exists:}

\begin{lstlisting}
Test-Path C:\tools\CIP-tool\bin\rgwm_versie2.4.3\bin\rgwm.exe
\end{lstlisting}
\item
  \textbf{Verify DIMR scripts folder exists:}

\begin{lstlisting}
Test-Path C:\tools\CIP-tool\bin\dimr_runners\sobek_dimr\x64\dimr\scripts
\end{lstlisting}
\end{enumerate}

Both commands should return \passthrough{\lstinline!True!}.

\begin{center}\rule{0.5\linewidth}{0.5pt}\end{center}

\subsection{Verifying Installation}\label{verifying-installation}

\subsubsection{Test CIP-tool Commands}\label{test-cip-tool-commands}

\begin{enumerate}
\def\labelenumi{\arabic{enumi}.}
\item
  \textbf{Check version:}

\begin{lstlisting}
cip-tool --version
\end{lstlisting}
\item
  \textbf{View available commands:}

\begin{lstlisting}
cip-tool --help
\end{lstlisting}
\end{enumerate}

\subsubsection{Run a Test Scenario}\label{run-a-test-scenario}

If you have the complete examples folder, you can test the full
workflow:

\begin{lstlisting}
# Generate boundary conditions
cip-tool generate -i examples/inputs/invoer_CIP_tabel.csv -o output

# Run Sobek simulation 
cip-tool run-sobek -o output

# extract results
cip-tool extract -i examples/inputs/variabelen_voor_bc_file_deelmodel.csv -o examples/extract-results/rmm_results
\end{lstlisting}

\begin{center}\rule{0.5\linewidth}{0.5pt}\end{center}

\subsection{Directory Structure
Reference}\label{directory-structure-reference}

Your working project structure should follow this pattern:

\begin{lstlisting}
your-project/
|-- config.yaml              # Configuration file (points to bin/ folder)
|-- inputs/
|   |-- invoer_CIP_tabel.csv              # Input scenarios
|   `-- variabelen_voor_bc_file_deelmodel.csv  # Variables config
|-- output/                  # Output directory (created automatically)
|   |-- rgwm_results/
|   `-- rmm_results/
`-- test_model/              # Your Sobek model (if running simulations)
    |-- dimr.xml
    `-- dflow1d/
\end{lstlisting}

The \passthrough{\lstinline!bin/!} folder can remain in the extraction
location (e.g., \passthrough{\lstinline!C:\\tools\\CIP-tool\\bin\\!})
and be referenced from multiple projects via their
\passthrough{\lstinline!config.yaml!} files.

\begin{center}\rule{0.5\linewidth}{0.5pt}\end{center}

\subsection{Next Steps}\label{next-steps}

After installation:

\begin{enumerate}
\def\labelenumi{\arabic{enumi}.}
\tightlist
\item
  \textbf{Read the next chapter
  \passthrough{\lstinline!Command Line Interface!}} to learn how to use
  CIP-tool commands
\item
  \textbf{Explore the examples folder} to understand the input file
  formats
\item
  \textbf{Prepare your own scenario data} following the example formats
\item
  \textbf{Start using CIP-tool} with your models
\end{enumerate}

\begin{center}\rule{0.5\linewidth}{0.5pt}\end{center}

\subsection{Troubleshooting}\label{troubleshooting}

\subsubsection{Python Not Found}\label{python-not-found}

If you get ``python is not recognized'':

\begin{enumerate}
\def\labelenumi{\arabic{enumi}.}
\tightlist
\item
  Make sure Python 3.11 is installed
\item
  Add Python to your PATH environment variable
\item
  Try using \passthrough{\lstinline!py -3.11!} instead of
  \passthrough{\lstinline!python!}
\end{enumerate}

\subsubsection{Wheel Installation Fails}\label{wheel-installation-fails}

If pip fails to install the wheel:

\begin{enumerate}
\def\labelenumi{\arabic{enumi}.}
\tightlist
\item
  Ensure you have pip installed:
  \passthrough{\lstinline!python -m ensurepip --upgrade!}
\item
  Try upgrading pip:
  \passthrough{\lstinline!python -m pip install --upgrade pip!}
\item
  Use the full pip command:
  \passthrough{\lstinline!python -m pip install CIP\_tool-1.1.0-py3-none-any.whl!}
\end{enumerate}

\subsubsection{External Tools Not Found}\label{external-tools-not-found}

If CIP-tool can't find RGWM or DIMR:

\begin{enumerate}
\def\labelenumi{\arabic{enumi}.}
\tightlist
\item
  Verify the paths in your \passthrough{\lstinline!config.yaml!} are
  correct
\item
  Use absolute paths in the config file
\item
  Make sure the executables exist at the specified locations
\item
  Check file permissions (executables should be runnable)
\end{enumerate}

\subsubsection{Permission Issues}\label{permission-issues}

If you get permission errors:

\begin{enumerate}
\def\labelenumi{\arabic{enumi}.}
\tightlist
\item
  Run PowerShell as Administrator
\item
  Or install to a user directory:
  \passthrough{\lstinline!pip install --user CIP\_tool-1.1.0-py3-none-any.whl!}
\end{enumerate}

\begin{center}\rule{0.5\linewidth}{0.5pt}\end{center}

\subsection{Need Help?}\label{need-help}

For additional support:

\begin{itemize}
\tightlist
\item
  Contact support: software.support@deltares.nl
\end{itemize}

\begin{center}\rule{0.5\linewidth}{0.5pt}\end{center}

\subsection{Updating CIP-tool}\label{updating-cip-tool}

To update to a newer version:

\begin{enumerate}
\def\labelenumi{\arabic{enumi}.}
\item
  Uninstall the current version:

\begin{lstlisting}
pip uninstall cip-tool
\end{lstlisting}
\item
  Install the new wheel file:

\begin{lstlisting}
pip install CIP_tool-X.X.X-py3-none-any.whl
\end{lstlisting}
\item
  Update the \passthrough{\lstinline!bin/!} folder if new versions of
  external tools are included
\end{enumerate}

\begin{quote}
\textbf{Note:} Your configuration files and project data will not be
affected by the update.
\end{quote}
