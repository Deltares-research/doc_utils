% Pandoc compatibility commands
\providecommand{\tightlist}{%
  \setlength{\itemsep}{0pt}\setlength{\parskip}{0pt}%
}
\providecommand{\passthrough}[1]{#1}

%------------------------------------------------------------------------------
% Pandoc compatibility commands
%------------------------------------------------------------------------------
\providecommand{\tightlist}{%
  \setlength{\itemsep}{0pt}\setlength{\parskip}{0pt}%
}
\providecommand{\passthrough}[1]{#1}


%------------------------------------------------------------------------------
% Content (no document structure)
%------------------------------------------------------------------------------
\section{Command Line Interface (CLI)}\label{command-line-interface-cli}

The CIP-tool provides a powerful command-line interface for generating,
running, and extracting boundary conditions for hydraulic models. This
guide covers all available commands, their options, and detailed usage
examples.

\subsection{Overview}\label{overview}

The CIP-tool CLI consists of three main commands:

\begin{enumerate}
\def\labelenumi{\arabic{enumi}.}
\tightlist
\item
  \textbf{\passthrough{\lstinline!generate!}} - Generate boundary
  conditions from input scenarios
\item
  \textbf{\passthrough{\lstinline!run-sobek!}} - Execute Sobek hydraulic
  model simulations
\item
  \textbf{\passthrough{\lstinline!extract!}} - Process and extract
  boundary conditions for a nested model
\end{enumerate}

\subsubsection{General Usage}\label{general-usage}

\begin{lstlisting}[language=bash]
cip-tool [--version] [--help] <command> [<args>]
\end{lstlisting}

\subsubsection{Getting Help}\label{getting-help}

View general help:

\begin{lstlisting}[language=bash]
cip-tool --help
\end{lstlisting}

View help for a specific command:

\begin{lstlisting}[language=bash]
cip-tool generate --help
cip-tool run-sobek --help
cip-tool extract --help
\end{lstlisting}

Check version:

\begin{lstlisting}[language=bash]
cip-tool --version
\end{lstlisting}

\begin{center}\rule{0.5\linewidth}{0.5pt}\end{center}

\subsection{Commands}\label{commands}

\subsubsection{1. Generate Command}\label{generate-command}

The \passthrough{\lstinline!generate!} command creates boundary
conditions for hydraulic models based on input scenario data. The
CIP-tool is configured to create boundary conditions for a 1D SOBEK
model for the Rhine-Meuse estuary (\textbf{sobek-rmm-j15\_5-v4}). Under
the hood, the CIP-tool creates the boundary conditions with the
Randvoorwaarden Generator Water Modellen (\textbf{RGWM}). It sets all
the input files for the RGWM and executes it.

Typically, the scenarios to consider are so-called `Conditionele
Illustratie Punten' (CIP's). These are the scenarios that have to be
studied in the context of permitting for projects in the Dutch main
rivers. The tool can also be used to create other scenarios.

\paragraph{Syntax}\label{syntax}

\begin{lstlisting}[language=bash]
cip-tool generate -i INPUT_FILE -o OUTPUT_DIR [-c CONFIG_FILE] [--calculate-surge]
\end{lstlisting}

\paragraph{Arguments}\label{arguments}

\begin{longtable}[]{@{}
  >{\raggedright\arraybackslash}p{(\linewidth - 8\tabcolsep) * \real{0.16}}
  >{\raggedright\arraybackslash}p{(\linewidth - 8\tabcolsep) * \real{0.07}}
  >{\raggedright\arraybackslash}p{(\linewidth - 8\tabcolsep) * \real{0.22}}
  >{\raggedright\arraybackslash}p{(\linewidth - 8\tabcolsep) * \real{0.10}}
  >{\raggedright\arraybackslash}p{(\linewidth - 8\tabcolsep) * \real{0.35}}@{}}
\toprule\noalign{}
\begin{minipage}[b]{\linewidth}\raggedright
Argument
\end{minipage} & \begin{minipage}[b]{\linewidth}\raggedright
Short
\end{minipage} & \begin{minipage}[b]{\linewidth}\raggedright
Long
\end{minipage} & \begin{minipage}[b]{\linewidth}\raggedright
Required
\end{minipage} & \begin{minipage}[b]{\linewidth}\raggedright
Description
\end{minipage} \\
\midrule\noalign{}
\endhead
\bottomrule\noalign{}
\endlastfoot
Input file & \passthrough{\lstinline!-i!} &
\passthrough{\lstinline!--input-file!} & Yes & Path to the CSV file
containing scenario data \\
Output directory & \passthrough{\lstinline!-o!} &
\passthrough{\lstinline!--output-dir!} & Yes & Path to output directory
(created if it doesn't exist) \\
Config file & \passthrough{\lstinline!-c!} &
\passthrough{\lstinline!--config!} & No & Path to custom config.yaml
file \\
Calculate surge & & \passthrough{\lstinline!--calculate-surge!} & No &
Enable iterative boundary condition generation \\
\end{longtable}

\paragraph{Input File Format}\label{input-file-format}

The scenarios to consider are configured in a CSV input file. This file
must contain the following columns, each row of the file contains the
inut for one scenario:

\begin{longtable}[]{@{}
  >{\raggedright\arraybackslash}p{(\linewidth - 6\tabcolsep) * \real{0.28}}
  >{\raggedright\arraybackslash}p{(\linewidth - 6\tabcolsep) * \real{0.30}}
  >{\raggedright\arraybackslash}p{(\linewidth - 6\tabcolsep) * \real{0.12}}
  >{\raggedright\arraybackslash}p{(\linewidth - 6\tabcolsep) * \real{0.30}}@{}}
\toprule\noalign{}
\begin{minipage}[b]{\linewidth}\raggedright
Column
\end{minipage} & \begin{minipage}[b]{\linewidth}\raggedright
Description
\end{minipage} & \begin{minipage}[b]{\linewidth}\raggedright
Unit
\end{minipage} & \begin{minipage}[b]{\linewidth}\raggedright
Required
\end{minipage} \\
\midrule\noalign{}
\endhead
\bottomrule\noalign{}
\endlastfoot
Scenario & Scenario identifier/number & - & Yes \\
\seqsplit{Keringstoestand[}open/gesloten{]} & Barrier state (open/closed) & - &
Yes \\
Rijnafvoer te Lobith{[}m3/s{]} & Rhine discharge at Lobith & m³/s &
Yes \\
Maasafvoer te Lith{[}m3/s{]} & Meuse discharge at Lith & m³/s & Yes \\
\seqsplit{Stormopzet[}m + NAP{]} & Maximum storm surge & m + NAP & Only without
\passthrough{\lstinline!--calculate-surge!} \\
\seqsplit{Zeewaterstand[}m + NAP{]} & Maximum total sea water level & m + NAP &
Only with \passthrough{\lstinline!--calculate-surge!} \\
\seqsplit{Windsnelheid[}m/s{]} & Wind speed & m/s & Yes \\
\seqsplit{Windrichting[}-{]} & Wind direction & - & Yes \\
\end{longtable}

\textbf{Example input file
(\passthrough{\lstinline!invoer\_CIP\_tabel.csv!}):}

\begin{lstlisting}
Scenario,Keringstoestand[open/gesloten],Rijnafvoer te Lobith[m3/s],Maasafvoer te Lith[m3/s],Stormopzet[m + NAP],Zeewaterstand[m + NAP],Windsnelheid[m/s],Windrichting[-]
1,Open,9000,1861,3,5,19.38,WNW
2,Gesloten,11500,2448,4.5,6,27.41,WNW
\end{lstlisting}

\paragraph{\texorpdfstring{Using the \texttt{-\/-calculate-surge}
flag}{Using the -\/-calculate-surge flag}}\label{using-the---calculate-surge-flag}

To generate the correct waterlevel timeseries at the downstream
boundaries, the user should choose oe of two options: 1. Enter the
maximum amount of storm surge in the column
\passthrough{\lstinline!Stormopzet[m + NAP]!} and run the
\passthrough{\lstinline!generate!} command \textbf{without} the flag
\passthrough{\lstinline!--calculate-surge!}. The resulting downstream
water level for each scenario is the sum of astronomical tides, sea
level rise and the amount of storm surge entered in the input file. The
storm surge is variable in time and reaches the set maximum. 2. Enter
the maximum total water level at the Maasmondin the column
\passthrough{\lstinline!Zeewaterstand[m + NAP]!} and the run the
\passthrough{\lstinline!generate!} command \textbf{with} the flag
\passthrough{\lstinline!--calculate-surge!}. The CIP tool will now
calculate the amount of max storm surge that together with astronomical
tides and sea level rise reaches the expected maximum water level. This
option should be taken when creating scenarios for set CIP's in
permitting projects.

When \passthrough{\lstinline!--calculate-surge!} is enabled:

\begin{enumerate}
\def\labelenumi{\arabic{enumi}.}
\tightlist
\item
  The tool performs iterative simulations to find the correct storm
  surge values
\item
  Each iteration adjusts the surge to match target water levels
\item
  The process continues until convergence (typically 3-5 iterations)
\item
  Final adjusted surge values are saved for the model runs
\item
  The column \passthrough{\lstinline!Zeewaterstand[m + NAP]!} becomes
  \textbf{required} in the input file
\end{enumerate}

\paragraph{Output}\label{output}

The command creates the following outputs in the specified directory:

\begin{itemize}
\tightlist
\item
  \passthrough{\lstinline!rgwm\_results/!} - RGWM model input files and
  scenarios
\item
  \passthrough{\lstinline!rmm\_results/!} - RMM model setup, including
  boundary conditions and batch files to run the models.
\item
  Scenario-specific folders (e.g., \passthrough{\lstinline!Q0/!},
  \passthrough{\lstinline!Q1/!}, etc.)
\item
  Log files and configuration files
\end{itemize}

\paragraph{Examples}\label{examples}

\subparagraph{Basic Generation (Default
Configuration)}\label{basic-generation-default-configuration}

Generate boundary conditions using default tool paths:

\begin{lstlisting}[language=bash]
cip-tool generate -i examples/inputs/invoer_CIP_tabel.csv -o examples/output
\end{lstlisting}

\textbf{Output:}

\begin{lstlisting}
Done.
\end{lstlisting}

\subparagraph{Generation with Custom
Configuration}\label{generation-with-custom-configuration}

Use a custom configuration file to specify tool paths:

\begin{lstlisting}[language=bash]
cip-tool generate -i examples/inputs/invoer_CIP_tabel.csv -o examples/output -c examples/config.yaml
\end{lstlisting}

\subparagraph{Generation with Surge
Calculation}\label{generation-with-surge-calculation}

Enable iterative surge calculation to achieve target water levels:

\begin{lstlisting}[language=bash]
cip-tool generate -i examples/inputs/invoer_CIP_tabel.csv -o examples/output --calculate-surge
\end{lstlisting}

\textbf{Output:}

\begin{lstlisting}
Iteratie 1
Done.
Maximale waterstand in scenario Q0: 3.71 m +NAP
Verschil met beoogde CIP maximale waterstand: -1.29 m + NAP
Maximale waterstand in scenario Q1: 5.14 m +NAP
Verschil met beoogde CIP maximale waterstand: -0.86 m + NAP
Iteratie 2
Done.
Maximale waterstand in scenario Q0: 4.94 m +NAP
Verschil met beoogde CIP maximale waterstand: -0.06 m + NAP
Maximale waterstand in scenario Q1: 5.98 m +NAP
Verschil met beoogde CIP maximale waterstand: -0.02 m + NAP
Iteratie 3
Done.
Maximale waterstand in scenario Q0: 4.99 m +NAP
Verschil met beoogde CIP maximale waterstand: -0.01 m + NAP
Maximale waterstand in scenario Q1: 6.0 m +NAP
Verschil met beoogde CIP maximale waterstand: 0.0 m + NAP
Iteratie 4
Done.
Maximale waterstand in scenario Q0: 5.0 m +NAP
Verschil met beoogde CIP maximale waterstand: 0.0 m + NAP
Maximale waterstand in scenario Q1: 6.0 m +NAP
Verschil met beoogde CIP maximale waterstand: 0.0 m + NAP
Geconvergeerd na 4 iteraties

Beoogde CIP condities:
   Scenario  Zeewaterstand[m + NAP]
0         1                     5.0
1         2                     6.0

Condities voor RGWM:
   Scenario  Stormopzet[m + NAP]
0         1                 4.36
1         2                 5.38

Verschillen:
[0. 0.]
\end{lstlisting}

\begin{center}\rule{0.5\linewidth}{0.5pt}\end{center}

\subsubsection{2. Run-Sobek Command}\label{run-sobek-command}

The \passthrough{\lstinline!run-sobek!} command executes Sobek hydraulic
model batch jobs and streams output to both the console and a log file.
For scenarios with closing storm surge barriers, the Singlerunner is
employed to account for correct operation of the storm surge barriers.

\paragraph{Syntax}\label{syntax-1}

\begin{lstlisting}[language=bash]
cip-tool run-sobek (-o OUTPUT_DIR | -b BATCH_FILE)
\end{lstlisting}

\paragraph{Arguments}\label{arguments-1}

You must provide \textbf{exactly one} of the following:

\begin{longtable}[]{@{}
  >{\raggedright\arraybackslash}p{(\linewidth - 6\tabcolsep) * \real{0.25}}
  >{\raggedright\arraybackslash}p{(\linewidth - 6\tabcolsep) * \real{0.09}}
  >{\raggedright\arraybackslash}p{(\linewidth - 6\tabcolsep) * \real{0.22}}
  >{\raggedright\arraybackslash}p{(\linewidth - 6\tabcolsep) * \real{0.44}}@{}}
\toprule\noalign{}
\begin{minipage}[b]{\linewidth}\raggedright
Argument
\end{minipage} & \begin{minipage}[b]{\linewidth}\raggedright
Short
\end{minipage} & \begin{minipage}[b]{\linewidth}\raggedright
Long
\end{minipage} & \begin{minipage}[b]{\linewidth}\raggedright
Description
\end{minipage} \\
\midrule\noalign{}
\endhead
\bottomrule\noalign{}
\endlastfoot
Output directory & \passthrough{\lstinline!-o!} &
\passthrough{\lstinline!--output-dir!} & Path to output directory in
which the models were prepare using the
\passthrough{\lstinline!generate!} command earlier (the run script is
auto-resolved within this file) \\
Batch file & \passthrough{\lstinline!-b!} &
\passthrough{\lstinline!--batch-file!} & Direct path to the run bscript
atch (.bat) file \\
\end{longtable}

\paragraph{System Requirements}\label{system-requirements}

\begin{itemize}
\tightlist
\item
  \textbf{Windows only} - Batch files require Windows operating system
\item
  Sobek/DIMR must be properly installed
\item
  Required environment variables should be set
\end{itemize}

\paragraph{Behavior}\label{behavior}

\begin{itemize}
\tightlist
\item
  Executes the Sobek batch file through Windows command processor
\item
  Streams all output to the console in real-time
\item
  Saves complete log to a \passthrough{\lstinline!.log!} file next to
  the batch file
\item
  Returns the exit code of the batch process
\item
  Handles paths with spaces correctly
\end{itemize}

\paragraph{Examples}\label{examples-1}

\subparagraph{Run with Output
Directory}\label{run-with-output-directory}

Automatically locate and run the batch file in the RMM results folder:

\begin{lstlisting}[language=bash]
cip-tool run-sobek -o examples/output
\end{lstlisting}

The tool looks for the batch file at:

\begin{lstlisting}
examples/output/rmm_results/bsub_joblist.bat
\end{lstlisting}

\textbf{Output:}

\begin{lstlisting}
C:\...\examples\output\rmm_results>set OMP_NUM_THREADS=2

C:\...\examples\output\rmm_results>start ''[started in Q0]'' /d ''Q0/model_output'' /w /b run_sobekdimr.bat

C:\...\examples\output\rmm_results\Q0\model_output>C:\...\dimr\scripts\run_dimr.bat dimr.xml
Configfile:dimr.xml
OMP_NUM_THREADS is already defined
OMP_NUM_THREADS is 2
Working directory: C:\...\examples\output\rmm_results\Q0\model_output
executing: ''C:\...\dimr\bin\dimr.exe''  dimr.xml
Dimr [2025-12-08 14:30:24.920] #0 >> Deltares, DIMR_EXE Version 2.00.00.68167M
Dimr [2025-12-08 14:30:24.987] #0 >> Deltares, DIMR_LIB Version 1.02.00.68167M
...
[Simulation progress output]
...
\end{lstlisting}

\subparagraph{Run with Explicit Batch
File}\label{run-with-explicit-batch-file}

Directly execute a specific batch file:

\begin{lstlisting}[language=bash]
cip-tool run-sobek -b examples/output/rmm_results/bsub_joblist.bat
\end{lstlisting}

\subparagraph{Log File Location}\label{log-file-location}

The log file is automatically created alongside the batch file:

\begin{itemize}
\tightlist
\item
  Batch file:
  \passthrough{\lstinline!examples/output/rmm\_results/bsub\_joblist.bat!}
\item
  Log file:
  \passthrough{\lstinline!examples/output/rmm\_results/bsub\_joblist.log!}
\end{itemize}

\paragraph{Error Handling}\label{error-handling}

If the batch execution fails (non-zero exit code), the command will exit
with the same error code:

\begin{lstlisting}[language=bash]
cip-tool run-sobek -o examples/output
# Returns non-zero exit code on failure
\end{lstlisting}

\begin{center}\rule{0.5\linewidth}{0.5pt}\end{center}

\subsubsection{3. Extract Command}\label{extract-command}

The \passthrough{\lstinline!extract!} command processes simulation
results and extracts boundary condition data for a nested model. By
default, the CIP-tool is configured to generate boundary conditions for
the 2D D-HYDRO model of the Biesbosch. Boundary conditions that are
prepared for the model are:

\begin{itemize}
\tightlist
\item
  Wind speed and direction
\item
  Waterlevels at downstream boundariers
\item
  River discharges at upstream boundaries
\end{itemize}

Lateral discharges are not created, however these can be re-used from
the SOBEK model boundary conditions that were generated with the
\passthrough{\lstinline!generate!} command.

\paragraph{Syntax}\label{syntax-2}

\begin{lstlisting}[language=bash]
cip-tool extract -i INPUT_FILE -o OUTPUT_DIR [-c CONFIG_FILE]
\end{lstlisting}

\paragraph{Arguments}\label{arguments-2}

\begin{longtable}[]{@{}
  >{\raggedright\arraybackslash}p{(\linewidth - 8\tabcolsep) * \real{0.19}}
  >{\raggedright\arraybackslash}p{(\linewidth - 8\tabcolsep) * \real{0.08}}
  >{\raggedright\arraybackslash}p{(\linewidth - 8\tabcolsep) * \real{0.18}}
  >{\raggedright\arraybackslash}p{(\linewidth - 8\tabcolsep) * \real{0.12}}
  >{\raggedright\arraybackslash}p{(\linewidth - 8\tabcolsep) * \real{0.33}}@{}}
\toprule\noalign{}
\begin{minipage}[b]{\linewidth}\raggedright
Argument
\end{minipage} & \begin{minipage}[b]{\linewidth}\raggedright
Short
\end{minipage} & \begin{minipage}[b]{\linewidth}\raggedright
Long
\end{minipage} & \begin{minipage}[b]{\linewidth}\raggedright
Required
\end{minipage} & \begin{minipage}[b]{\linewidth}\raggedright
Description
\end{minipage} \\
\midrule\noalign{}
\endhead
\bottomrule\noalign{}
\endlastfoot
Input file & \passthrough{\lstinline!-i!} &
\passthrough{\lstinline!--input-file!} & Yes & Path to extraction
configuration CSV \\
Output directory & \passthrough{\lstinline!-o!} &
\passthrough{\lstinline!--output-dir!} & Yes & Path to directory
containing simulation results \\
Config file & \passthrough{\lstinline!-c!} &
\passthrough{\lstinline!--config!} & No & Path to custom config.yaml
file \\
\end{longtable}

\paragraph{Input File Format}\label{input-file-format-1}

The extraction configuration CSV must contain the following columns:

\begin{longtable}[]{@{}lll@{}}
\toprule\noalign{}
Column & Description & Example \\
\midrule\noalign{}
\endhead
\bottomrule\noalign{}
\endlastfoot
file & Name of the NetCDF result file &
\passthrough{\lstinline!observations.nc!} \\
location & Location identifier in the model &
\passthrough{\lstinline!BE\_976.00!} \\
new\_name\_loc & New name for the location &
\passthrough{\lstinline!Beneden-Merwede\_0001!} \\
variabele & Variable name to extract &
\passthrough{\lstinline!water\_level!} \\
new\_name\_var & New name for the variable &
\passthrough{\lstinline!waterlevelbnd!} \\
\end{longtable}

\textbf{Example extraction file
(\passthrough{\lstinline!variabelen\_voor\_bc\_file\_deelmodel.csv!}):}

\begin{lstlisting}
file,location,new_name_loc,variabele,new_name_var
observations.nc,BE_976.00,Beneden-Merwede_0001,water_level,waterlevelbnd
observations.nc,WL_935.00,Waal_0001,water_discharge,dischargebnd
observations.nc,HD_983.00,Nieuwe-Merwede_0001,water_level,waterlevelbnd
observations.nc,MA_230.00,Maas_0001,water_discharge,dischargebnd
\end{lstlisting}

\paragraph{Output}\label{output-1}

The command extracts data from simulation results and creates:

\begin{itemize}
\tightlist
\item
  Processed boundary condition files
\item
  Renamed and reformatted data for nested models
\end{itemize}

\paragraph{Examples}\label{examples-2}

\subparagraph{Basic Extraction}\label{basic-extraction}

Extract results

\begin{lstlisting}[language=bash]
cip-tool extract -i examples/inputs/variabelen_voor_bc_file_deelmodel.csv -o examples/extract-results/rmm_results
\end{lstlisting}

\textbf{Output:}

\begin{lstlisting}
File : observations.nc Locatie : BE_976.00 Variabel : water_level
File : observations.nc Locatie : WL_935.00 Variabel : water_discharge
File : observations.nc Locatie : HD_983.00 Variabel : water_level
File : observations.nc Locatie : MA_230.00 Variabel : water_discharge
Done.
\end{lstlisting}

\subparagraph{Extracting from Specific Result
Directory}\label{extracting-from-specific-result-directory}

If results are in a subdirectory:

\begin{lstlisting}[language=bash]
cip-tool extract -i examples/inputs/variabelen_voor_bc_file_deelmodel.csv -o examples/output/rmm_results
\end{lstlisting}

\begin{center}\rule{0.5\linewidth}{0.5pt}\end{center}

\subsection{Configuration}\label{configuration}

\subsubsection{Default Configuration}\label{default-configuration}

When the \passthrough{\lstinline!--config!} argument is \textbf{not}
specified, the tool uses default paths relative to the current
directory:

\begin{itemize}
\tightlist
\item
  \textbf{DIMR path:}
  \passthrough{\lstinline!bin/dimr\_runners/sobek\_dimr/x64/dimr/scripts!}
\item
  \textbf{RGWM path:}
  \passthrough{\lstinline!bin/rgwm\_versie2.4.3/bin/rgwm.exe!}
\end{itemize}

\subsubsection{Custom Configuration}\label{custom-configuration}

You can provide a custom \passthrough{\lstinline!config.yaml!} file to
override default paths.

\paragraph{Config File Structure}\label{config-file-structure}

\begin{lstlisting}
DIMR:
    path: bin/dimr_runners/sobek_dimr/x64/dimr/scripts

RGWM:
    path: bin/rgwm_versie2.4.3/bin/rgwm.exe
\end{lstlisting}

\paragraph{Configuration Requirements}\label{configuration-requirements}

\begin{itemize}
\tightlist
\item
  \textbf{Both sections required:} DIMR and RGWM paths must be specified
\item
  \textbf{DIMR path:} Must point to an existing directory
\item
  \textbf{RGWM path:} Must point to an executable file (rgwm.exe)
\item
  \textbf{Path types:} Can be absolute or relative to the config file
  location
\end{itemize}

\paragraph{Example Configuration
Files}\label{example-configuration-files}

\textbf{Absolute paths:}

\begin{lstlisting}
DIMR:
    path: C:/tools/sobek/dimr/scripts

RGWM:
    path: C:/tools/rgwm/bin/rgwm.exe
\end{lstlisting}

\textbf{Relative paths (relative to config.yaml location):}

\begin{lstlisting}
DIMR:
    path: ../bin/dimr_runners/sobek_dimr/x64/dimr/scripts

RGWM:
    path: ../bin/rgwm_versie2.4.3/bin/rgwm.exe
\end{lstlisting}

\begin{center}\rule{0.5\linewidth}{0.5pt}\end{center}

\subsection{Workflow Example}\label{workflow-example}

Here's a complete workflow from scenario generation to result
extraction:

\subsubsection{Step 1: Prepare Input
Files}\label{step-1-prepare-input-files}

Create your scenario input file:

\begin{lstlisting}
Scenario,Keringstoestand[open/gesloten],Rijnafvoer te Lobith[m3/s],Maasafvoer te Lith[m3/s],Stormopzet[m + NAP],Zeewaterstand[m + NAP],Windsnelheid[m/s],Windrichting[-]
1,Open,9000,1861,3,5,19.38,WNW
2,Gesloten,11500,2448,4.5,6,27.41,WNW
3,Open,12000,2600,5,6.5,30.5,W
\end{lstlisting}

\subsubsection{Step 2: Generate Boundary
Conditions}\label{step-2-generate-boundary-conditions}

\begin{lstlisting}[language=bash]
cip-tool generate -i scenarios.csv -o output --calculate-surge
\end{lstlisting}

\subsubsection{Step 3: Run Sobek
Simulations}\label{step-3-run-sobek-simulations}

\begin{lstlisting}[language=bash]
cip-tool run-sobek -o output
\end{lstlisting}

\subsubsection{Step 4: Extract Results}\label{step-4-extract-results}

\begin{lstlisting}[language=bash]
cip-tool extract -i extraction_config.csv -o output
\end{lstlisting}

\begin{center}\rule{0.5\linewidth}{0.5pt}\end{center}

\subsection{Troubleshooting}\label{troubleshooting}

\subsubsection{Common Issues}\label{common-issues}

\paragraph{Input File Not Found}\label{input-file-not-found}

\textbf{Error:}

\begin{lstlisting}
ArgumentTypeError: 'input.csv' is not an existing file.
\end{lstlisting}

\textbf{Solution:} - Verify the file path is correct - Use absolute
paths or ensure relative paths are correct - Check file permissions

\paragraph{Output Directory Creation
Failed}\label{output-directory-creation-failed}

\textbf{Error:}

\begin{lstlisting}
PermissionError: [Errno 13] Permission denied
\end{lstlisting}

\textbf{Solution:} - Check write permissions on the target directory -
Ensure the path is valid - Try running with elevated privileges if
necessary

\paragraph{Batch File Not Found
(run-sobek)}\label{batch-file-not-found-run-sobek}

\textbf{Error:}

\begin{lstlisting}
FileNotFoundError: Batch file not found: ...
\end{lstlisting}

\textbf{Solution:} - Ensure you've run the
\passthrough{\lstinline!generate!} command first - Verify the output
directory contains the \passthrough{\lstinline!rmm\_results!} folder -
Use absolute paths or check your current working directory

\paragraph{Wrong Operating System
(run-sobek)}\label{wrong-operating-system-run-sobek}

\textbf{Error:}

\begin{lstlisting}
RuntimeError: Running a .bat file requires Windows.
\end{lstlisting}

\textbf{Solution:} - The \passthrough{\lstinline!run-sobek!} command
only works on Windows - Use a Windows machine or WSL with Windows
interop enabled

\paragraph{Missing Configuration
Paths}\label{missing-configuration-paths}

\textbf{Error:}

\begin{lstlisting}
FileNotFoundError: DIMR path does not exist
\end{lstlisting}

\textbf{Solution:} - Verify paths in config.yaml are correct - Ensure
DIMR and RGWM are properly installed - Check that paths are relative to
the config file location

\paragraph{Surge Calculation Not
Converging}\label{surge-calculation-not-converging}

\textbf{Issue:} Iterative surge calculation takes too many iterations or
doesn't converge

\textbf{Solution:} - Check that input water levels are realistic -
Verify model setup is correct - Review initial storm surge estimates -
Check for model stability issues

\begin{center}\rule{0.5\linewidth}{0.5pt}\end{center}

\subsection{Best Practices}\label{best-practices}

\subsubsection{File Organization}\label{file-organization}

\begin{lstlisting}
project/
|-- config.yaml
|-- inputs/
|   |-- scenarios.csv
|   `-- extraction_config.csv
|-- output/
|   |-- rgwm_results/
|   `-- rmm_results/
`-- bin/
    |-- dimr_runners/
    `-- rgwm_versie2.4.3/
\end{lstlisting}

\subsubsection{Tips}\label{tips}

\begin{enumerate}
\def\labelenumi{\arabic{enumi}.}
\tightlist
\item
  \textbf{Use version control} for input files and configuration
\item
  \textbf{Create template files} for consistent scenario definitions
\item
  \textbf{Document your scenarios} with meaningful names/numbers
\item
  \textbf{Back up results} before re-running simulations
\item
  \textbf{Use meaningful output directory names} (e.g.,
  \passthrough{\lstinline!output\_2025\_flood\_study!})
\item
  \textbf{Check logs} after Sobek runs for warnings or errors
\item
  \textbf{Validate extracted data} before using in downstream analyses
\end{enumerate}

\subsubsection{Performance
Considerations}\label{performance-considerations}

\begin{itemize}
\tightlist
\item
  Large scenario sets may take considerable time to run
\item
  Surge calculation adds multiple iterations (typically 3-5x runtime)
\item
  Consider parallel execution for independent scenarios
\item
  Monitor disk space for large result files
\end{itemize}

\begin{center}\rule{0.5\linewidth}{0.5pt}\end{center}

\subsection{Additional Resources}\label{additional-resources}

\begin{itemize}
\tightlist
\item
  \href{../api/cli-quick-reference.md}{Quick Reference Guide}
\item
  \href{installation-zip-file.md}{Installation Guide}
\item
  \href{../api/cli.md}{API Documentation}
\item
  \href{../change-log.md}{Change Log}
\end{itemize}

\begin{center}\rule{0.5\linewidth}{0.5pt}\end{center}

\subsection{Support}\label{support}

For issues, questions, or contributions:

\begin{itemize}
\tightlist
\item
  Check the \href{../index.md}{documentation}
\item
  Review the \href{installation-zip-file.md}{installation guide}
\item
  Contact support: software.support@deltares.nl
\end{itemize}
